% -------[ DEFINIZIONI DI STILE PERSONALI ]-----------------------------------------
% -------[ Edoardo Porcaro ]--------------------------------------------------------
% -------[ Iniziato il 11 agosto 2025 ]---------------------------------------------
% -------[ Ultimo aggiornamento il 17 settembre 2025 ]------------------------------
% -------[ v 1.3 ]------------------------------------------------------------------

% -------[ COME USARE IL PACCHETTO ]------------------------------------------------
%   1. Subito sotto a \documentclass{***}, incollare la seguente riga
%       \PassOptionsToClass{11pt,a4paper,twoside,titlepage}{***}
%           oppure riscrivere \documentclass{***} come
%       \documentclass[11pt,a4paper,twoside,titlepage]{***}
%   2. Caricare il pacchetto dopo \documentclass{***} con
%       % -------[ DEFINIZIONI DI STILE PERSONALI ]-----------------------------------------
% -------[ Edoardo Porcaro ]--------------------------------------------------------
% -------[ Iniziato il 11 agosto 2025 ]---------------------------------------------
% -------[ Ultimo aggiornamento il 17 settembre 2025 ]------------------------------
% -------[ v 1.3 ]------------------------------------------------------------------

% -------[ COME USARE IL PACCHETTO ]------------------------------------------------
%   1. Subito sotto a \documentclass{***}, incollare la seguente riga
%       \PassOptionsToClass{11pt,a4paper,twoside,titlepage}{***}
%           oppure riscrivere \documentclass{***} come
%       \documentclass[11pt,a4paper,twoside,titlepage]{***}
%   2. Caricare il pacchetto dopo \documentclass{***} con
%       % -------[ DEFINIZIONI DI STILE PERSONALI ]-----------------------------------------
% -------[ Edoardo Porcaro ]--------------------------------------------------------
% -------[ Iniziato il 11 agosto 2025 ]---------------------------------------------
% -------[ Ultimo aggiornamento il 17 settembre 2025 ]------------------------------
% -------[ v 1.3 ]------------------------------------------------------------------

% -------[ COME USARE IL PACCHETTO ]------------------------------------------------
%   1. Subito sotto a \documentclass{***}, incollare la seguente riga
%       \PassOptionsToClass{11pt,a4paper,twoside,titlepage}{***}
%           oppure riscrivere \documentclass{***} come
%       \documentclass[11pt,a4paper,twoside,titlepage]{***}
%   2. Caricare il pacchetto dopo \documentclass{***} con
%       % -------[ DEFINIZIONI DI STILE PERSONALI ]-----------------------------------------
% -------[ Edoardo Porcaro ]--------------------------------------------------------
% -------[ Iniziato il 11 agosto 2025 ]---------------------------------------------
% -------[ Ultimo aggiornamento il 17 settembre 2025 ]------------------------------
% -------[ v 1.3 ]------------------------------------------------------------------

% -------[ COME USARE IL PACCHETTO ]------------------------------------------------
%   1. Subito sotto a \documentclass{***}, incollare la seguente riga
%       \PassOptionsToClass{11pt,a4paper,twoside,titlepage}{***}
%           oppure riscrivere \documentclass{***} come
%       \documentclass[11pt,a4paper,twoside,titlepage]{***}
%   2. Caricare il pacchetto dopo \documentclass{***} con
%       \input{porcaro}
%   3. In questo file, andare alla riga
%       pdftitle = {***},
%      e sostituire a *** l'effettivo titolo del documento,
%      affinché il metadato "Titolo" del pdf sia aggiornato correttamente
% -------[ fine ]-------------------------------------------------------------------

% -------[ INDICE DEL PACCHETTO ]---------------------------------------------------
%   IMPOSTAZIONI INIZIALI
%      Definizione dell'autore (mio nome e mail)
%      Impostazione pagina, testo e caratteri
%      Matematica
%      Figure e grafici
%   PERSONALIZZAZIONI
%      Personalizzazione didascalie
%      Personalizzazione intestazioni e piè di pagina
%      Collegamenti e segnalibri
%      Personalizzazione riferimenti
%      Personalizzazione \section, \subsection e \subsubsection
%    DEFINIZIONI PERSONALIZZATE
%      Definizione \esempio
%      Definizione \osservazione
%      Definizioni matematiche personalizzate
% -------[ fine ]-------------------------------------------------------------------

% ----------------------------------------------------------------------------------
% -------[ INIZIO CONTENUTO PACCHETTO ]---------------------------------------------
% -------[ Non toccare se non si capisce cosa c'è scritto ]-------------------------
% ----------------------------------------------------------------------------------

% -------[ DEFINIZIONE DELL'AUTORE (MIO NOME E MAIL) ]------------------------------
    \newcommand{\Autore}{Edoardo~Porcaro}
    \newcommand{\EdoardoPorcaro}{\href{mailto:ufogamespa@gmail.com}{\Autore{}}}
% -------[ fine ]-------------------------------------------------------------------

% -------[ IMPOSTAZIONE PAGINA, TESTO E CARATTERI ]---------------------------------
    \usepackage[utf8]{inputenc} % Codifica input UTF8
    \usepackage[T1]{fontenc} % Codifica output T1 (supporto a lettere accentate europee)
    \usepackage[italian]{babel} % Usa la lingua italiana (sillabazione e nomenclatura sezioni, indici e etichette (es. "figura"))
        \usepackage{silence}\WarningFilter{latex}{Command \showhyphens has changed} % Dalla versione 2025 di pdfLaTeX esce questo warning che, con questo pacchetto e comando, nascondo dal log
    \addto\captionsitalian{
        \renewcommand{\abstractname}{Descrizione del corso} % Rinomina il "nome" dell'abstract
        %\renewcommand{\listfigurename}{Indice delle figure} % Rinomina il "nome" del listoffigures
        \renewcommand{\contentsname}{Indice dei contenuti}} % Rinomina il "nome" del tableofcontents
    \usepackage{microtype} % Rende la tipografia più professionale e raffinata
    \usepackage{multicol} % Per consentire un fine controllo su pagine e sezioni multicolonna
    \usepackage{titlesec} % Usato per ridefinire le \subsubsection nel blocco [Personalizzazione \subsubsection]
    \usepackage{enumitem} % Consente di personalizzare gli elenchi puntati o numerati (listati), modificandone margini, spaziatura, simboli, numerazioni
    \usepackage{pifont} % Per numeretti cerchiati (es. \ding{172} = (1))
    \usepackage{bbding} % Per \PencilRight in \es e \HandRight in \oss
    %\setlength{\parindent}{0pt} % Rimuovi l'indentazione a inizio nuova riga
    \setcounter{secnumdepth}{3} % Numerazione fino al livello \subsubsection [2]
    \setcounter{tocdepth}{3} % Metti nel toc fino al livello \subsubsection [3]
    \frenchspacing % Usa la spaziatura europea dopo i segni di punteggiatura
% -------[ fine ]-------------------------------------------------------------------

% -------[ MATEMATICA ]-------------------------------------------------------------
    \usepackage{amsmath,amssymb,amsfonts,amsthm} % Pacchetti standard dell'AMS
        \numberwithin{equation}{section} % Per avere la numerazione delle equazioni che segua quella delle \section
    \usepackage{mathtools} % Fornisce comandi aggiuntivi per l’allineamento e la formattazione di equazioni (es. \coloneqq)
    \usepackage{mathrsfs} % Per lettere calligrafiche "più corsive" (es. \mathscr{C})
    \usepackage{bbm} % Per rendere qualsiasi carattere (numeri inclusi!) in stile blackboard bold (es. \mathbbm{R} -- che in questo caso è uguale a \mathbb{R})
    \usepackage{cancel}
    \usepackage{stmaryrd} % Introduce simboli matematici aggiuntivi non presenti negli standard LaTeX o AMS (es. \llbracket e \rrbracket)
    \usepackage{siunitx}
% -------[ fine ]-------------------------------------------------------------------

% -------[ FIGURE E GRAFICI ]-------------------------------------------------------
    \usepackage{graphicx}
    \usepackage{tikz}
        \usetikzlibrary{arrows.meta,positioning} % Per opzioni avanzate per le punte delle frecce e per il posizionamento relativo dei nodi
        \usetikzlibrary{angles,quotes} % Per disegnare angoli ed etichette
        \usetikzlibrary{patterns} % Per "escludere" aree del piano con i pattern
        \usetikzlibrary{calc} % Per effettuare calcoli dentro le coordinate
    % Definizione di elementi asse, tratteggio e tratteggio scuro per tutti i grafici TikZ
        \tikzset{
            asse/.style = {-{Stealth[length=6pt,width=6pt]}},
            punto/.style = {circle, fill=black, inner sep=1.2pt},
            punto/.style = {circle, fill=black, minimum size=1.2pt},
            tratteggio/.style = {dashed, gray},
            tratteggioscuro/.style = {dashed, black}
        }
    % Definizioni di stile per i grafici con pgfplots e l'ambiente axis
    \usepackage{pgfplots}
        \pgfplotsset{compat=1.18}

        % Stile di default per tutti gli axis
        \pgfplotsset{
            every axis/.style={
                axis lines=middle,
                xlabel={$x$}, ylabel={$y$},
                xmin=-5, xmax=5,
                ymin=-5, ymax=5,
                xtick={0}, ytick={0},
                unbounded coords=jump,
                enlargelimits=false,
            },
        % Stile di default per tutti gli \addplot
            every axis plot/.style={
                thick,
                smooth,
                samples=200
            }
        }
    % Definizione di punto (come comando stand alone: la definizione in tikzset non si scala correttamente)
        \newcommand{\punto}[1]{\filldraw (#1) circle (1.2pt)}
% -------[ fine ]-------------------------------------------------------------------

% -------[ PERSONALIZZAZIONE DIDASCALIE ]-------------------------------------------
    \usepackage{caption}
        \captionsetup{
            labelfont = sf, % Etichetta (es. "Figura 1") in corsivo 
            labelsep = period, % Separatore "." tra etichetta e testo (es. "Figura 1. Lorem ipsum") (di default è ":")
            textfont = normalfont,% testo normale (puoi mettere it per corsivo)
            %justification=raggedright, % evita giustificazione completa
            %singlelinecheck=false, % applica lo stile anche se la didascalia è su una sola riga
            margin = 1.5cm % Stringi specularmente i margini di 1,5cm di tutta la didascalia
        }
    \usepackage{subcaption}
% -------[ fine ]-------------------------------------------------------------------

% -------[ PERSONALIZZAZIONE INTESTAZIONI E PIÈ DI PAGINA ]-------------------------
    \usepackage{fancyhdr}
    % Carica i nomi di section/subsection da usare nelle intestazioni
        \renewcommand{\sectionmark}[1]{\markboth{\thesection~\,#1}{}}
        \renewcommand{\subsectionmark}[1]{\markright{\thesubsection~\,#1}}
    % Pulisce intestazione e piè di pagina (per modificarli a piacimento)
        \fancyhf{}
    % Imposta l'altezza dell'intestazione affinché sia sufficiente (>13,5 pt)
        \setlength{\headheight}{14pt}
    % Nessuna linea orizzontale né sotto all'intestazione né sopra al piè di pagina
        \renewcommand{\headrulewidth}{0pt}
        \renewcommand{\footrulewidth}{0pt}
    % Definisci lo stile di intestazioni e piè di pagina predefinito
        \fancypagestyle{predefinito}{
            % Lato esterno sinistro (Left Even): "[section]"
                \fancyhead[LE]{\sffamily\nouppercase{\leftmark}}
            % Lato esterno destro (Right Odd): "[subsection]"
                \fancyhead[RO]{\sffamily\nouppercase{\rightmark}}
            % Lati interni (Left Odd e Right Even): "Edoardo Porcaro"
                \fancyhead[LO,RE]{\EdoardoPorcaro{}}
            % Numeri di pagina nei lati esterni del piè di pagina (a sinistra nelle pagine pari e a destra nelle pagine dispari)
                \fancyfoot[LE,RO]{\sffamily\thepage}
        }
    % Definisci lo stile di intestazioni e piè di pagina per le pagine in cui si apre una \section
        \fancypagestyle{inizio-section}{
            % Svuoto i lati esterni sinistro e destro affinché non appaiano indicazioni su sezione o sottosezione
            %   Il resto lo lascio invariato, ossia come definito dallo stile "predefinito"
                \fancyhead[LE,RO]{}
        }
    % Imposta lo stile di intestazioni e piè di pagina predefinito per tutto il documento (salvo eccezioni definite pagina per pagina nel corpo del documento che sovrascrivono lo stile predefinito)
        \pagestyle{predefinito}
% -------[ fine ]-------------------------------------------------------------------

% -------[ COLLEGAMENTI E SEGNALIBRI ]----------------------------------------------
    \usepackage{hyperref}
    % Imposta le proprietà (i metadati) del documento PDF
        \hypersetup{
            linktoc = all, % Pure i numeri di pagina nel toc sono cliccabili
            hidelinks,
            pdftitle = {Riassunto di analisi matematica 1},
            pdfauthor = {\Autore{}},
            pdfsubject = {Analisi matematica 1},
            %pdfkeywords = {keywords},
            pdfcreator = {\Autore{}},
            pdfproducer = {Computer di \Autore{}},
            %pdfpagetransition = {Wipe /Di 270 /SS 0.5}
            pdfdisplaydoctitle = true % Mostra il titolo (pdftitle) del documento nel titolo della finestra del lettore PDF (anzichè il nome del file)
        }
    % Rendi tutti i collegamenti (\href) in sans serif
        \let\oldhref\href
        \renewcommand{\href}[2]{\sffamily\oldhref{#1}{#2}}
    % Metti nei segnalibri del PDF tutti i livelli fino al livello \paragraph [4]
    \usepackage{bookmark}
        \bookmarksetup{depth=4}
% -------[ fine ]-------------------------------------------------------------------

% -------[ PERSONALIZZAZIONE RIFERIMENTI ]------------------------------------------
    \usepackage{varioref} % Per usare \vref, che indica anche la pagina se l'elemento a cui punta non è sulla stessa pagina del comando
    \usepackage{cleveref} % Per usare \cref, che indica anche l'etichetta dell'elemento a cui punta (ad esempio, scrive automaticamente "fig." oltre suo numero)
    %   Traduzioni italiane delle congiunzioni
        \newcommand{\crefpairconjunction}{ e } % Tra due referenze
        \newcommand{\crefmiddleconjunction}{, } % Tra referenze intermedie
        \newcommand{\creflastconjunction}{ e } % Prima dell’ultima referenza
    % Tutti i riferimenti resi in sans serif
    %   Equazioni (eq., eqq., Eq. e Eqq.) (tolgo anche le parentesti altrimenti presenti)
        \creflabelformat{equation}{\sffamily#1}
        % Minuscola
            \crefname{equation}{\sffamily eq.}{\sffamily eqq.}
            \crefrangeformat{equation}{\sffamily eqq. da #1 a #2}
        % Maiuscola
            \Crefname{equation}{Eq.}{Eqq.}
            \Crefrangeformat{equation}{\sffamily Eqq. da #1 a #2}
    %   Figure (fig. e Fig.)
        \creflabelformat{figure}{\sffamily#2#1#3}
            \crefformat{figure}{\sffamily#2fig.~#1#3} % Con la minuscola
            \Crefformat{figure}{\sffamily#2Fig.~#1#3} % Con la maiuscola
    %   Tabelle (tab. e Tab.)
        \crefformat{table}{\sffamily#2tab.~#1#3} % Con la minuscola
        \Crefformat{table}{\sffamily#2Tab.~#1#3} % Con la maiuscola
    %   \section (cap. e Cap.)
        \crefformat{section}{\sffamily#2cap.~#1#3} % Con la minuscola
        \Crefformat{section}{\sffamily#2Cap.~#1#3} % Con la maiuscola
    %   \subsection (sez. e Sez.)
        \crefformat{subsection}{\sffamily#2sez.~#1#3} % Con la minuscola
        \Crefformat{subsection}{\sffamily#2Sez.~#1#3} % Con la maiuscola
    %   \subsubsection (par. e Par.)
        \crefformat{subsubsection}{\sffamily#2par.~#1#3} % Con la minuscola
        \Crefformat{subsubsection}{\sffamily#2Par.~#1#3} % Con la maiuscola
    %   Pagina
        \crefformat{page}{\sffamily#2pag.~#1#3}
        \Crefformat{page}{\sffamily#2Pag.~#1#3}
    % Per riferirsi sia all'oggetto sia alla pagina in cui è presente
        \newcommand{\crefpage}[1]{\cref{#1} \textsf{a} \cpageref{#1}}
% -------[ fine ]-------------------------------------------------------------------

% -------[ PERSONALIZZAZIONE \SECTION, \SUBSECTION E \SUBSUBSECTION ]---------------
    % Definisci il nuovo controllo condizionale \iffirstsubsection per verificare se è la prima subsection della section a cui appartiene
        \newif\iffirstsubsection
            \firstsubsectiontrue
    % Riga orizzontale di separazione (HOrizontal rule)
        \newcommand{\horule}{\noindent\rule{\linewidth}{0.4pt}}
    % \SECTION
        % Ridefinisci le \section affinché si aprano sempre a pagina nuova e, nella pagina in cui si aprono, non ci sia l'intestazione relativa a sezione o sottosezione (lati esterni sinistro e destro)
        \let\ssection\section
        \renewcommand{\section}{
            \clearpage
            \thispagestyle{inizio-section}
            \firstsubsectiontrue % Dico che è iniziata una nuova section, affinché la prima subsection non abbia l'\horule
            \ssection
        }
    % \SUBSECTION
        % Ridefinisci le \subsection affinché, fuorché la prima di ogni section, abbiano tutte la riga di separazione (\horule)
        \let\ssubsection\subsection
        \renewcommand{\subsection}{
            \iffirstsubsection % Se è la prima subsection dopo una nuova section,
                \firstsubsectionfalse % allora avvisa che la prossima sarà la seconda (e adesso non mettere l'\horule
            \else % Se non è la prima subsection dopo una nuova section,
                \horule % allora stampa l'\horule
            \fi
            \ssubsection
        }
    % \SUBSUBSECTION
        % Ridefinisci le \subsubsection affinché siano stilisticamente come dei paragrafi
        \titleformat{\subsubsection}[runin] % [runin] affinché il testo che segue stia sulla stessa riga del titolo della subsection
            {\normalfont\normalsize\bfseries} % Stile del titolo
            {} % Etichetta (niente numero o simbolo prima del titolo)
            {0pt} % Spazio tra l’etichetta e il titolo
            {} % Codice da eseguire prima del titolo (nulla)
            [.] % Codice da eseguire dopo il titolo (metti il punto)
% -------[ fine ]-------------------------------------------------------------------

% -------[ DEFINIZIONE \ESEMPIO ]---------------------------------------------------
    % Manina (per mettere in risalto gli esempi nel testo)
    \newcommand{\Matita}{
        \normalfont\hspace{-0.25em}\raisebox{-0.3em}[0pt][0pt]{\scalebox{1.2}[1.1]{\PencilRight}}\hspace{0.1em}
    }
    % Definisci lo stile specifico del paragrafo "Esempio" (matita)
    %   Comando: \es
    %   \es*[opzionale]{obbligatorio}
    %   *           :  indica se includere l'esempio nel toc e nei segnalibri PDF
    %   opzionale   :  indica il titolo dell'esempio per il toc e i segnalibri PDF
    %   obbligatorio:  indica il titolo dell'esempio visibile nel PDF
        \NewDocumentCommand{\es}{s o m}{
            \IfBooleanTF{#1} {
            %   \es*{lorem}
                {\paragraph*{\Matita{}\itshape #3.}}
            }{
            %   \es{lorem}
                \IfNoValueTF{#2}
                    {\paragraph[#3]{\Matita{}\itshape #3.}}
            %   \es[ipsum]{lorem}
                {\paragraph[#2]{\Matita{}\itshape #3.}}
            }
        }
    % Comando per finire graficamente l'esempio (con l'inserimento di \PencilLeft)
    %   \FineEs       se l'esempio finisce in mathmode (non inline)
    %   \FineEsTesto  se l'esempio finisce con una normale riga di testo
        \newcommand{\FineEs}{\tag*{\raisebox{-0.2em}[0pt][0pt]{\scalebox{1}[1]{\PencilLeft}}}}
        \newcommand{\FineEsTesto}{\hfill\raisebox{-0.2em}[0pt][0pt]{\scalebox{1}[1]{\PencilLeft}}}
% -------[ fine ]-------------------------------------------------------------------

% -------[ DEFINIZIONE \OSSERVAZIONE ]----------------------------------------------
    % Definisci lo stile specifico del paragrafo "Osservazione/Nota bene" (manina)
    %   Comandi: \oss \osservazione \nb \notabene
        \newcommand{\oss}[1]{
            \paragraph[Osservazione]{\normalfont % No grassetto
                \raisebox{-0.1em}[0pt][0pt]{\scalebox{1.2}[1]{\HandRight}}\hspace{-0.4em} % Manina
                {\itshape #1} % Eventuale titolo dell'osservazione tutto in corsivo (punto escluso)
            }
        }
        \newcommand{\osservazione}[1]{\oss{#1}}
        \newcommand{\nb}[1]{\oss{#1}}
        \newcommand{\notabene}[1]{\oss{#1}}
% -------[ fine ]-------------------------------------------------------------------

% -------[ DEFINIZIONI MATEMATICHE PERSONALIZZATE ]---------------------------------
    \newcommand{\co}[2]{\left(#1\,,#2\right)} % Coppia Ordinata \co12 => (1 , 2)
    \renewcommand{\i}{\mathrm i} % Unità immaginaria \i
    \renewcommand{\Re}{\operatorname{Re}} % Parte reale dei numeri complessi \Re
    \renewcommand{\Im}{\operatorname{Im}} % Parte immaginaria dei numeri compessi \Im
    \newcommand{\an}{\left\{a_n\right\}} % Successione a_n \an
    \renewcommand{\d}{\textrm d} % Differeziale d
    
    % Insiemistica
        \newcommand{\Set}[1]{\left\{ #1 \right\}} % Rappresentazione d'insieme per elencazione \Set{1,2,3} => {1,2,3}
        \newcommand{\N}{\mathbb{N}} % Insieme dei numeri naturali
        \newcommand{\Z}{\mathbb{Z}} % Insieme dei numeri relativi
        \newcommand{\Q}{\mathbb{Q}} % Insieme dei numeri razionali
        \newcommand{\R}{\mathbb{R}} % Insieme dei numeri reali
        \newcommand{\C}{\mathbb{C}} % Insieme dei numeri complessi
    
    % Lettere greche preferite (inverto quella di default e quella alternativa)
        % epsilon
            \let\altepsilon\epsilon % ALTernativa (sulla base delle mie pref.)
            \let\corepsilon\varepsilon % CORretta (sulla base delle mie pref.)
            \renewcommand{\epsilon}{\corepsilon}
            \renewcommand{\varepsilon}{\altepsilon}
        % phi
            \let\altphi\phi % ALTernativa (sulla base delle mie pref.)
            \let\corphi\varphi % CORretta (sulla base delle mie pref.)
            \renewcommand{\phi}{\corphi}
            \renewcommand{\varphi}{\altphi}
% -------[ fine ]-------------------------------------------------------------------
%   3. In questo file, andare alla riga
%       pdftitle = {***},
%      e sostituire a *** l'effettivo titolo del documento,
%      affinché il metadato "Titolo" del pdf sia aggiornato correttamente
% -------[ fine ]-------------------------------------------------------------------

% -------[ INDICE DEL PACCHETTO ]---------------------------------------------------
%   IMPOSTAZIONI INIZIALI
%      Definizione dell'autore (mio nome e mail)
%      Impostazione pagina, testo e caratteri
%      Matematica
%      Figure e grafici
%   PERSONALIZZAZIONI
%      Personalizzazione didascalie
%      Personalizzazione intestazioni e piè di pagina
%      Collegamenti e segnalibri
%      Personalizzazione riferimenti
%      Personalizzazione \section, \subsection e \subsubsection
%    DEFINIZIONI PERSONALIZZATE
%      Definizione \esempio
%      Definizione \osservazione
%      Definizioni matematiche personalizzate
% -------[ fine ]-------------------------------------------------------------------

% ----------------------------------------------------------------------------------
% -------[ INIZIO CONTENUTO PACCHETTO ]---------------------------------------------
% -------[ Non toccare se non si capisce cosa c'è scritto ]-------------------------
% ----------------------------------------------------------------------------------

% -------[ DEFINIZIONE DELL'AUTORE (MIO NOME E MAIL) ]------------------------------
    \newcommand{\Autore}{Edoardo~Porcaro}
    \newcommand{\EdoardoPorcaro}{\href{mailto:ufogamespa@gmail.com}{\Autore{}}}
% -------[ fine ]-------------------------------------------------------------------

% -------[ IMPOSTAZIONE PAGINA, TESTO E CARATTERI ]---------------------------------
    \usepackage[utf8]{inputenc} % Codifica input UTF8
    \usepackage[T1]{fontenc} % Codifica output T1 (supporto a lettere accentate europee)
    \usepackage[italian]{babel} % Usa la lingua italiana (sillabazione e nomenclatura sezioni, indici e etichette (es. "figura"))
        \usepackage{silence}\WarningFilter{latex}{Command \showhyphens has changed} % Dalla versione 2025 di pdfLaTeX esce questo warning che, con questo pacchetto e comando, nascondo dal log
    \addto\captionsitalian{
        \renewcommand{\abstractname}{Descrizione del corso} % Rinomina il "nome" dell'abstract
        %\renewcommand{\listfigurename}{Indice delle figure} % Rinomina il "nome" del listoffigures
        \renewcommand{\contentsname}{Indice dei contenuti}} % Rinomina il "nome" del tableofcontents
    \usepackage{microtype} % Rende la tipografia più professionale e raffinata
    \usepackage{multicol} % Per consentire un fine controllo su pagine e sezioni multicolonna
    \usepackage{titlesec} % Usato per ridefinire le \subsubsection nel blocco [Personalizzazione \subsubsection]
    \usepackage{enumitem} % Consente di personalizzare gli elenchi puntati o numerati (listati), modificandone margini, spaziatura, simboli, numerazioni
    \usepackage{pifont} % Per numeretti cerchiati (es. \ding{172} = (1))
    \usepackage{bbding} % Per \PencilRight in \es e \HandRight in \oss
    %\setlength{\parindent}{0pt} % Rimuovi l'indentazione a inizio nuova riga
    \setcounter{secnumdepth}{3} % Numerazione fino al livello \subsubsection [2]
    \setcounter{tocdepth}{3} % Metti nel toc fino al livello \subsubsection [3]
    \frenchspacing % Usa la spaziatura europea dopo i segni di punteggiatura
% -------[ fine ]-------------------------------------------------------------------

% -------[ MATEMATICA ]-------------------------------------------------------------
    \usepackage{amsmath,amssymb,amsfonts,amsthm} % Pacchetti standard dell'AMS
        \numberwithin{equation}{section} % Per avere la numerazione delle equazioni che segua quella delle \section
    \usepackage{mathtools} % Fornisce comandi aggiuntivi per l’allineamento e la formattazione di equazioni (es. \coloneqq)
    \usepackage{mathrsfs} % Per lettere calligrafiche "più corsive" (es. \mathscr{C})
    \usepackage{bbm} % Per rendere qualsiasi carattere (numeri inclusi!) in stile blackboard bold (es. \mathbbm{R} -- che in questo caso è uguale a \mathbb{R})
    \usepackage{cancel}
    \usepackage{stmaryrd} % Introduce simboli matematici aggiuntivi non presenti negli standard LaTeX o AMS (es. \llbracket e \rrbracket)
    \usepackage{siunitx}
% -------[ fine ]-------------------------------------------------------------------

% -------[ FIGURE E GRAFICI ]-------------------------------------------------------
    \usepackage{graphicx}
    \usepackage{tikz}
        \usetikzlibrary{arrows.meta,positioning} % Per opzioni avanzate per le punte delle frecce e per il posizionamento relativo dei nodi
        \usetikzlibrary{angles,quotes} % Per disegnare angoli ed etichette
        \usetikzlibrary{patterns} % Per "escludere" aree del piano con i pattern
        \usetikzlibrary{calc} % Per effettuare calcoli dentro le coordinate
    % Definizione di elementi asse, tratteggio e tratteggio scuro per tutti i grafici TikZ
        \tikzset{
            asse/.style = {-{Stealth[length=6pt,width=6pt]}},
            punto/.style = {circle, fill=black, inner sep=1.2pt},
            punto/.style = {circle, fill=black, minimum size=1.2pt},
            tratteggio/.style = {dashed, gray},
            tratteggioscuro/.style = {dashed, black}
        }
    % Definizioni di stile per i grafici con pgfplots e l'ambiente axis
    \usepackage{pgfplots}
        \pgfplotsset{compat=1.18}

        % Stile di default per tutti gli axis
        \pgfplotsset{
            every axis/.style={
                axis lines=middle,
                xlabel={$x$}, ylabel={$y$},
                xmin=-5, xmax=5,
                ymin=-5, ymax=5,
                xtick={0}, ytick={0},
                unbounded coords=jump,
                enlargelimits=false,
            },
        % Stile di default per tutti gli \addplot
            every axis plot/.style={
                thick,
                smooth,
                samples=200
            }
        }
    % Definizione di punto (come comando stand alone: la definizione in tikzset non si scala correttamente)
        \newcommand{\punto}[1]{\filldraw (#1) circle (1.2pt)}
% -------[ fine ]-------------------------------------------------------------------

% -------[ PERSONALIZZAZIONE DIDASCALIE ]-------------------------------------------
    \usepackage{caption}
        \captionsetup{
            labelfont = sf, % Etichetta (es. "Figura 1") in corsivo 
            labelsep = period, % Separatore "." tra etichetta e testo (es. "Figura 1. Lorem ipsum") (di default è ":")
            textfont = normalfont,% testo normale (puoi mettere it per corsivo)
            %justification=raggedright, % evita giustificazione completa
            %singlelinecheck=false, % applica lo stile anche se la didascalia è su una sola riga
            margin = 1.5cm % Stringi specularmente i margini di 1,5cm di tutta la didascalia
        }
    \usepackage{subcaption}
% -------[ fine ]-------------------------------------------------------------------

% -------[ PERSONALIZZAZIONE INTESTAZIONI E PIÈ DI PAGINA ]-------------------------
    \usepackage{fancyhdr}
    % Carica i nomi di section/subsection da usare nelle intestazioni
        \renewcommand{\sectionmark}[1]{\markboth{\thesection~\,#1}{}}
        \renewcommand{\subsectionmark}[1]{\markright{\thesubsection~\,#1}}
    % Pulisce intestazione e piè di pagina (per modificarli a piacimento)
        \fancyhf{}
    % Imposta l'altezza dell'intestazione affinché sia sufficiente (>13,5 pt)
        \setlength{\headheight}{14pt}
    % Nessuna linea orizzontale né sotto all'intestazione né sopra al piè di pagina
        \renewcommand{\headrulewidth}{0pt}
        \renewcommand{\footrulewidth}{0pt}
    % Definisci lo stile di intestazioni e piè di pagina predefinito
        \fancypagestyle{predefinito}{
            % Lato esterno sinistro (Left Even): "[section]"
                \fancyhead[LE]{\sffamily\nouppercase{\leftmark}}
            % Lato esterno destro (Right Odd): "[subsection]"
                \fancyhead[RO]{\sffamily\nouppercase{\rightmark}}
            % Lati interni (Left Odd e Right Even): "Edoardo Porcaro"
                \fancyhead[LO,RE]{\EdoardoPorcaro{}}
            % Numeri di pagina nei lati esterni del piè di pagina (a sinistra nelle pagine pari e a destra nelle pagine dispari)
                \fancyfoot[LE,RO]{\sffamily\thepage}
        }
    % Definisci lo stile di intestazioni e piè di pagina per le pagine in cui si apre una \section
        \fancypagestyle{inizio-section}{
            % Svuoto i lati esterni sinistro e destro affinché non appaiano indicazioni su sezione o sottosezione
            %   Il resto lo lascio invariato, ossia come definito dallo stile "predefinito"
                \fancyhead[LE,RO]{}
        }
    % Imposta lo stile di intestazioni e piè di pagina predefinito per tutto il documento (salvo eccezioni definite pagina per pagina nel corpo del documento che sovrascrivono lo stile predefinito)
        \pagestyle{predefinito}
% -------[ fine ]-------------------------------------------------------------------

% -------[ COLLEGAMENTI E SEGNALIBRI ]----------------------------------------------
    \usepackage{hyperref}
    % Imposta le proprietà (i metadati) del documento PDF
        \hypersetup{
            linktoc = all, % Pure i numeri di pagina nel toc sono cliccabili
            hidelinks,
            pdftitle = {Riassunto di analisi matematica 1},
            pdfauthor = {\Autore{}},
            pdfsubject = {Analisi matematica 1},
            %pdfkeywords = {keywords},
            pdfcreator = {\Autore{}},
            pdfproducer = {Computer di \Autore{}},
            %pdfpagetransition = {Wipe /Di 270 /SS 0.5}
            pdfdisplaydoctitle = true % Mostra il titolo (pdftitle) del documento nel titolo della finestra del lettore PDF (anzichè il nome del file)
        }
    % Rendi tutti i collegamenti (\href) in sans serif
        \let\oldhref\href
        \renewcommand{\href}[2]{\sffamily\oldhref{#1}{#2}}
    % Metti nei segnalibri del PDF tutti i livelli fino al livello \paragraph [4]
    \usepackage{bookmark}
        \bookmarksetup{depth=4}
% -------[ fine ]-------------------------------------------------------------------

% -------[ PERSONALIZZAZIONE RIFERIMENTI ]------------------------------------------
    \usepackage{varioref} % Per usare \vref, che indica anche la pagina se l'elemento a cui punta non è sulla stessa pagina del comando
    \usepackage{cleveref} % Per usare \cref, che indica anche l'etichetta dell'elemento a cui punta (ad esempio, scrive automaticamente "fig." oltre suo numero)
    %   Traduzioni italiane delle congiunzioni
        \newcommand{\crefpairconjunction}{ e } % Tra due referenze
        \newcommand{\crefmiddleconjunction}{, } % Tra referenze intermedie
        \newcommand{\creflastconjunction}{ e } % Prima dell’ultima referenza
    % Tutti i riferimenti resi in sans serif
    %   Equazioni (eq., eqq., Eq. e Eqq.) (tolgo anche le parentesti altrimenti presenti)
        \creflabelformat{equation}{\sffamily#1}
        % Minuscola
            \crefname{equation}{\sffamily eq.}{\sffamily eqq.}
            \crefrangeformat{equation}{\sffamily eqq. da #1 a #2}
        % Maiuscola
            \Crefname{equation}{Eq.}{Eqq.}
            \Crefrangeformat{equation}{\sffamily Eqq. da #1 a #2}
    %   Figure (fig. e Fig.)
        \creflabelformat{figure}{\sffamily#2#1#3}
            \crefformat{figure}{\sffamily#2fig.~#1#3} % Con la minuscola
            \Crefformat{figure}{\sffamily#2Fig.~#1#3} % Con la maiuscola
    %   Tabelle (tab. e Tab.)
        \crefformat{table}{\sffamily#2tab.~#1#3} % Con la minuscola
        \Crefformat{table}{\sffamily#2Tab.~#1#3} % Con la maiuscola
    %   \section (cap. e Cap.)
        \crefformat{section}{\sffamily#2cap.~#1#3} % Con la minuscola
        \Crefformat{section}{\sffamily#2Cap.~#1#3} % Con la maiuscola
    %   \subsection (sez. e Sez.)
        \crefformat{subsection}{\sffamily#2sez.~#1#3} % Con la minuscola
        \Crefformat{subsection}{\sffamily#2Sez.~#1#3} % Con la maiuscola
    %   \subsubsection (par. e Par.)
        \crefformat{subsubsection}{\sffamily#2par.~#1#3} % Con la minuscola
        \Crefformat{subsubsection}{\sffamily#2Par.~#1#3} % Con la maiuscola
    %   Pagina
        \crefformat{page}{\sffamily#2pag.~#1#3}
        \Crefformat{page}{\sffamily#2Pag.~#1#3}
    % Per riferirsi sia all'oggetto sia alla pagina in cui è presente
        \newcommand{\crefpage}[1]{\cref{#1} \textsf{a} \cpageref{#1}}
% -------[ fine ]-------------------------------------------------------------------

% -------[ PERSONALIZZAZIONE \SECTION, \SUBSECTION E \SUBSUBSECTION ]---------------
    % Definisci il nuovo controllo condizionale \iffirstsubsection per verificare se è la prima subsection della section a cui appartiene
        \newif\iffirstsubsection
            \firstsubsectiontrue
    % Riga orizzontale di separazione (HOrizontal rule)
        \newcommand{\horule}{\noindent\rule{\linewidth}{0.4pt}}
    % \SECTION
        % Ridefinisci le \section affinché si aprano sempre a pagina nuova e, nella pagina in cui si aprono, non ci sia l'intestazione relativa a sezione o sottosezione (lati esterni sinistro e destro)
        \let\ssection\section
        \renewcommand{\section}{
            \clearpage
            \thispagestyle{inizio-section}
            \firstsubsectiontrue % Dico che è iniziata una nuova section, affinché la prima subsection non abbia l'\horule
            \ssection
        }
    % \SUBSECTION
        % Ridefinisci le \subsection affinché, fuorché la prima di ogni section, abbiano tutte la riga di separazione (\horule)
        \let\ssubsection\subsection
        \renewcommand{\subsection}{
            \iffirstsubsection % Se è la prima subsection dopo una nuova section,
                \firstsubsectionfalse % allora avvisa che la prossima sarà la seconda (e adesso non mettere l'\horule
            \else % Se non è la prima subsection dopo una nuova section,
                \horule % allora stampa l'\horule
            \fi
            \ssubsection
        }
    % \SUBSUBSECTION
        % Ridefinisci le \subsubsection affinché siano stilisticamente come dei paragrafi
        \titleformat{\subsubsection}[runin] % [runin] affinché il testo che segue stia sulla stessa riga del titolo della subsection
            {\normalfont\normalsize\bfseries} % Stile del titolo
            {} % Etichetta (niente numero o simbolo prima del titolo)
            {0pt} % Spazio tra l’etichetta e il titolo
            {} % Codice da eseguire prima del titolo (nulla)
            [.] % Codice da eseguire dopo il titolo (metti il punto)
% -------[ fine ]-------------------------------------------------------------------

% -------[ DEFINIZIONE \ESEMPIO ]---------------------------------------------------
    % Manina (per mettere in risalto gli esempi nel testo)
    \newcommand{\Matita}{
        \normalfont\hspace{-0.25em}\raisebox{-0.3em}[0pt][0pt]{\scalebox{1.2}[1.1]{\PencilRight}}\hspace{0.1em}
    }
    % Definisci lo stile specifico del paragrafo "Esempio" (matita)
    %   Comando: \es
    %   \es*[opzionale]{obbligatorio}
    %   *           :  indica se includere l'esempio nel toc e nei segnalibri PDF
    %   opzionale   :  indica il titolo dell'esempio per il toc e i segnalibri PDF
    %   obbligatorio:  indica il titolo dell'esempio visibile nel PDF
        \NewDocumentCommand{\es}{s o m}{
            \IfBooleanTF{#1} {
            %   \es*{lorem}
                {\paragraph*{\Matita{}\itshape #3.}}
            }{
            %   \es{lorem}
                \IfNoValueTF{#2}
                    {\paragraph[#3]{\Matita{}\itshape #3.}}
            %   \es[ipsum]{lorem}
                {\paragraph[#2]{\Matita{}\itshape #3.}}
            }
        }
    % Comando per finire graficamente l'esempio (con l'inserimento di \PencilLeft)
    %   \FineEs       se l'esempio finisce in mathmode (non inline)
    %   \FineEsTesto  se l'esempio finisce con una normale riga di testo
        \newcommand{\FineEs}{\tag*{\raisebox{-0.2em}[0pt][0pt]{\scalebox{1}[1]{\PencilLeft}}}}
        \newcommand{\FineEsTesto}{\hfill\raisebox{-0.2em}[0pt][0pt]{\scalebox{1}[1]{\PencilLeft}}}
% -------[ fine ]-------------------------------------------------------------------

% -------[ DEFINIZIONE \OSSERVAZIONE ]----------------------------------------------
    % Definisci lo stile specifico del paragrafo "Osservazione/Nota bene" (manina)
    %   Comandi: \oss \osservazione \nb \notabene
        \newcommand{\oss}[1]{
            \paragraph[Osservazione]{\normalfont % No grassetto
                \raisebox{-0.1em}[0pt][0pt]{\scalebox{1.2}[1]{\HandRight}}\hspace{-0.4em} % Manina
                {\itshape #1} % Eventuale titolo dell'osservazione tutto in corsivo (punto escluso)
            }
        }
        \newcommand{\osservazione}[1]{\oss{#1}}
        \newcommand{\nb}[1]{\oss{#1}}
        \newcommand{\notabene}[1]{\oss{#1}}
% -------[ fine ]-------------------------------------------------------------------

% -------[ DEFINIZIONI MATEMATICHE PERSONALIZZATE ]---------------------------------
    \newcommand{\co}[2]{\left(#1\,,#2\right)} % Coppia Ordinata \co12 => (1 , 2)
    \renewcommand{\i}{\mathrm i} % Unità immaginaria \i
    \renewcommand{\Re}{\operatorname{Re}} % Parte reale dei numeri complessi \Re
    \renewcommand{\Im}{\operatorname{Im}} % Parte immaginaria dei numeri compessi \Im
    \newcommand{\an}{\left\{a_n\right\}} % Successione a_n \an
    \renewcommand{\d}{\textrm d} % Differeziale d
    
    % Insiemistica
        \newcommand{\Set}[1]{\left\{ #1 \right\}} % Rappresentazione d'insieme per elencazione \Set{1,2,3} => {1,2,3}
        \newcommand{\N}{\mathbb{N}} % Insieme dei numeri naturali
        \newcommand{\Z}{\mathbb{Z}} % Insieme dei numeri relativi
        \newcommand{\Q}{\mathbb{Q}} % Insieme dei numeri razionali
        \newcommand{\R}{\mathbb{R}} % Insieme dei numeri reali
        \newcommand{\C}{\mathbb{C}} % Insieme dei numeri complessi
    
    % Lettere greche preferite (inverto quella di default e quella alternativa)
        % epsilon
            \let\altepsilon\epsilon % ALTernativa (sulla base delle mie pref.)
            \let\corepsilon\varepsilon % CORretta (sulla base delle mie pref.)
            \renewcommand{\epsilon}{\corepsilon}
            \renewcommand{\varepsilon}{\altepsilon}
        % phi
            \let\altphi\phi % ALTernativa (sulla base delle mie pref.)
            \let\corphi\varphi % CORretta (sulla base delle mie pref.)
            \renewcommand{\phi}{\corphi}
            \renewcommand{\varphi}{\altphi}
% -------[ fine ]-------------------------------------------------------------------
%   3. In questo file, andare alla riga
%       pdftitle = {***},
%      e sostituire a *** l'effettivo titolo del documento,
%      affinché il metadato "Titolo" del pdf sia aggiornato correttamente
% -------[ fine ]-------------------------------------------------------------------

% -------[ INDICE DEL PACCHETTO ]---------------------------------------------------
%   IMPOSTAZIONI INIZIALI
%      Definizione dell'autore (mio nome e mail)
%      Impostazione pagina, testo e caratteri
%      Matematica
%      Figure e grafici
%   PERSONALIZZAZIONI
%      Personalizzazione didascalie
%      Personalizzazione intestazioni e piè di pagina
%      Collegamenti e segnalibri
%      Personalizzazione riferimenti
%      Personalizzazione \section, \subsection e \subsubsection
%    DEFINIZIONI PERSONALIZZATE
%      Definizione \esempio
%      Definizione \osservazione
%      Definizioni matematiche personalizzate
% -------[ fine ]-------------------------------------------------------------------

% ----------------------------------------------------------------------------------
% -------[ INIZIO CONTENUTO PACCHETTO ]---------------------------------------------
% -------[ Non toccare se non si capisce cosa c'è scritto ]-------------------------
% ----------------------------------------------------------------------------------

% -------[ DEFINIZIONE DELL'AUTORE (MIO NOME E MAIL) ]------------------------------
    \newcommand{\Autore}{Edoardo~Porcaro}
    \newcommand{\EdoardoPorcaro}{\href{mailto:ufogamespa@gmail.com}{\Autore{}}}
% -------[ fine ]-------------------------------------------------------------------

% -------[ IMPOSTAZIONE PAGINA, TESTO E CARATTERI ]---------------------------------
    \usepackage[utf8]{inputenc} % Codifica input UTF8
    \usepackage[T1]{fontenc} % Codifica output T1 (supporto a lettere accentate europee)
    \usepackage[italian]{babel} % Usa la lingua italiana (sillabazione e nomenclatura sezioni, indici e etichette (es. "figura"))
        \usepackage{silence}\WarningFilter{latex}{Command \showhyphens has changed} % Dalla versione 2025 di pdfLaTeX esce questo warning che, con questo pacchetto e comando, nascondo dal log
    \addto\captionsitalian{
        \renewcommand{\abstractname}{Descrizione del corso} % Rinomina il "nome" dell'abstract
        %\renewcommand{\listfigurename}{Indice delle figure} % Rinomina il "nome" del listoffigures
        \renewcommand{\contentsname}{Indice dei contenuti}} % Rinomina il "nome" del tableofcontents
    \usepackage{microtype} % Rende la tipografia più professionale e raffinata
    \usepackage{multicol} % Per consentire un fine controllo su pagine e sezioni multicolonna
    \usepackage{titlesec} % Usato per ridefinire le \subsubsection nel blocco [Personalizzazione \subsubsection]
    \usepackage{enumitem} % Consente di personalizzare gli elenchi puntati o numerati (listati), modificandone margini, spaziatura, simboli, numerazioni
    \usepackage{pifont} % Per numeretti cerchiati (es. \ding{172} = (1))
    \usepackage{bbding} % Per \PencilRight in \es e \HandRight in \oss
    %\setlength{\parindent}{0pt} % Rimuovi l'indentazione a inizio nuova riga
    \setcounter{secnumdepth}{3} % Numerazione fino al livello \subsubsection [2]
    \setcounter{tocdepth}{3} % Metti nel toc fino al livello \subsubsection [3]
    \frenchspacing % Usa la spaziatura europea dopo i segni di punteggiatura
% -------[ fine ]-------------------------------------------------------------------

% -------[ MATEMATICA ]-------------------------------------------------------------
    \usepackage{amsmath,amssymb,amsfonts,amsthm} % Pacchetti standard dell'AMS
        \numberwithin{equation}{section} % Per avere la numerazione delle equazioni che segua quella delle \section
    \usepackage{mathtools} % Fornisce comandi aggiuntivi per l’allineamento e la formattazione di equazioni (es. \coloneqq)
    \usepackage{mathrsfs} % Per lettere calligrafiche "più corsive" (es. \mathscr{C})
    \usepackage{bbm} % Per rendere qualsiasi carattere (numeri inclusi!) in stile blackboard bold (es. \mathbbm{R} -- che in questo caso è uguale a \mathbb{R})
    \usepackage{cancel}
    \usepackage{stmaryrd} % Introduce simboli matematici aggiuntivi non presenti negli standard LaTeX o AMS (es. \llbracket e \rrbracket)
    \usepackage{siunitx}
% -------[ fine ]-------------------------------------------------------------------

% -------[ FIGURE E GRAFICI ]-------------------------------------------------------
    \usepackage{graphicx}
    \usepackage{tikz}
        \usetikzlibrary{arrows.meta,positioning} % Per opzioni avanzate per le punte delle frecce e per il posizionamento relativo dei nodi
        \usetikzlibrary{angles,quotes} % Per disegnare angoli ed etichette
        \usetikzlibrary{patterns} % Per "escludere" aree del piano con i pattern
        \usetikzlibrary{calc} % Per effettuare calcoli dentro le coordinate
    % Definizione di elementi asse, tratteggio e tratteggio scuro per tutti i grafici TikZ
        \tikzset{
            asse/.style = {-{Stealth[length=6pt,width=6pt]}},
            punto/.style = {circle, fill=black, inner sep=1.2pt},
            punto/.style = {circle, fill=black, minimum size=1.2pt},
            tratteggio/.style = {dashed, gray},
            tratteggioscuro/.style = {dashed, black}
        }
    % Definizioni di stile per i grafici con pgfplots e l'ambiente axis
    \usepackage{pgfplots}
        \pgfplotsset{compat=1.18}

        % Stile di default per tutti gli axis
        \pgfplotsset{
            every axis/.style={
                axis lines=middle,
                xlabel={$x$}, ylabel={$y$},
                xmin=-5, xmax=5,
                ymin=-5, ymax=5,
                xtick={0}, ytick={0},
                unbounded coords=jump,
                enlargelimits=false,
            },
        % Stile di default per tutti gli \addplot
            every axis plot/.style={
                thick,
                smooth,
                samples=200
            }
        }
    % Definizione di punto (come comando stand alone: la definizione in tikzset non si scala correttamente)
        \newcommand{\punto}[1]{\filldraw (#1) circle (1.2pt)}
% -------[ fine ]-------------------------------------------------------------------

% -------[ PERSONALIZZAZIONE DIDASCALIE ]-------------------------------------------
    \usepackage{caption}
        \captionsetup{
            labelfont = sf, % Etichetta (es. "Figura 1") in corsivo 
            labelsep = period, % Separatore "." tra etichetta e testo (es. "Figura 1. Lorem ipsum") (di default è ":")
            textfont = normalfont,% testo normale (puoi mettere it per corsivo)
            %justification=raggedright, % evita giustificazione completa
            %singlelinecheck=false, % applica lo stile anche se la didascalia è su una sola riga
            margin = 1.5cm % Stringi specularmente i margini di 1,5cm di tutta la didascalia
        }
    \usepackage{subcaption}
% -------[ fine ]-------------------------------------------------------------------

% -------[ PERSONALIZZAZIONE INTESTAZIONI E PIÈ DI PAGINA ]-------------------------
    \usepackage{fancyhdr}
    % Carica i nomi di section/subsection da usare nelle intestazioni
        \renewcommand{\sectionmark}[1]{\markboth{\thesection~\,#1}{}}
        \renewcommand{\subsectionmark}[1]{\markright{\thesubsection~\,#1}}
    % Pulisce intestazione e piè di pagina (per modificarli a piacimento)
        \fancyhf{}
    % Imposta l'altezza dell'intestazione affinché sia sufficiente (>13,5 pt)
        \setlength{\headheight}{14pt}
    % Nessuna linea orizzontale né sotto all'intestazione né sopra al piè di pagina
        \renewcommand{\headrulewidth}{0pt}
        \renewcommand{\footrulewidth}{0pt}
    % Definisci lo stile di intestazioni e piè di pagina predefinito
        \fancypagestyle{predefinito}{
            % Lato esterno sinistro (Left Even): "[section]"
                \fancyhead[LE]{\sffamily\nouppercase{\leftmark}}
            % Lato esterno destro (Right Odd): "[subsection]"
                \fancyhead[RO]{\sffamily\nouppercase{\rightmark}}
            % Lati interni (Left Odd e Right Even): "Edoardo Porcaro"
                \fancyhead[LO,RE]{\EdoardoPorcaro{}}
            % Numeri di pagina nei lati esterni del piè di pagina (a sinistra nelle pagine pari e a destra nelle pagine dispari)
                \fancyfoot[LE,RO]{\sffamily\thepage}
        }
    % Definisci lo stile di intestazioni e piè di pagina per le pagine in cui si apre una \section
        \fancypagestyle{inizio-section}{
            % Svuoto i lati esterni sinistro e destro affinché non appaiano indicazioni su sezione o sottosezione
            %   Il resto lo lascio invariato, ossia come definito dallo stile "predefinito"
                \fancyhead[LE,RO]{}
        }
    % Imposta lo stile di intestazioni e piè di pagina predefinito per tutto il documento (salvo eccezioni definite pagina per pagina nel corpo del documento che sovrascrivono lo stile predefinito)
        \pagestyle{predefinito}
% -------[ fine ]-------------------------------------------------------------------

% -------[ COLLEGAMENTI E SEGNALIBRI ]----------------------------------------------
    \usepackage{hyperref}
    % Imposta le proprietà (i metadati) del documento PDF
        \hypersetup{
            linktoc = all, % Pure i numeri di pagina nel toc sono cliccabili
            hidelinks,
            pdftitle = {Riassunto di analisi matematica 1},
            pdfauthor = {\Autore{}},
            pdfsubject = {Analisi matematica 1},
            %pdfkeywords = {keywords},
            pdfcreator = {\Autore{}},
            pdfproducer = {Computer di \Autore{}},
            %pdfpagetransition = {Wipe /Di 270 /SS 0.5}
            pdfdisplaydoctitle = true % Mostra il titolo (pdftitle) del documento nel titolo della finestra del lettore PDF (anzichè il nome del file)
        }
    % Rendi tutti i collegamenti (\href) in sans serif
        \let\oldhref\href
        \renewcommand{\href}[2]{\sffamily\oldhref{#1}{#2}}
    % Metti nei segnalibri del PDF tutti i livelli fino al livello \paragraph [4]
    \usepackage{bookmark}
        \bookmarksetup{depth=4}
% -------[ fine ]-------------------------------------------------------------------

% -------[ PERSONALIZZAZIONE RIFERIMENTI ]------------------------------------------
    \usepackage{varioref} % Per usare \vref, che indica anche la pagina se l'elemento a cui punta non è sulla stessa pagina del comando
    \usepackage{cleveref} % Per usare \cref, che indica anche l'etichetta dell'elemento a cui punta (ad esempio, scrive automaticamente "fig." oltre suo numero)
    %   Traduzioni italiane delle congiunzioni
        \newcommand{\crefpairconjunction}{ e } % Tra due referenze
        \newcommand{\crefmiddleconjunction}{, } % Tra referenze intermedie
        \newcommand{\creflastconjunction}{ e } % Prima dell’ultima referenza
    % Tutti i riferimenti resi in sans serif
    %   Equazioni (eq., eqq., Eq. e Eqq.) (tolgo anche le parentesti altrimenti presenti)
        \creflabelformat{equation}{\sffamily#1}
        % Minuscola
            \crefname{equation}{\sffamily eq.}{\sffamily eqq.}
            \crefrangeformat{equation}{\sffamily eqq. da #1 a #2}
        % Maiuscola
            \Crefname{equation}{Eq.}{Eqq.}
            \Crefrangeformat{equation}{\sffamily Eqq. da #1 a #2}
    %   Figure (fig. e Fig.)
        \creflabelformat{figure}{\sffamily#2#1#3}
            \crefformat{figure}{\sffamily#2fig.~#1#3} % Con la minuscola
            \Crefformat{figure}{\sffamily#2Fig.~#1#3} % Con la maiuscola
    %   Tabelle (tab. e Tab.)
        \crefformat{table}{\sffamily#2tab.~#1#3} % Con la minuscola
        \Crefformat{table}{\sffamily#2Tab.~#1#3} % Con la maiuscola
    %   \section (cap. e Cap.)
        \crefformat{section}{\sffamily#2cap.~#1#3} % Con la minuscola
        \Crefformat{section}{\sffamily#2Cap.~#1#3} % Con la maiuscola
    %   \subsection (sez. e Sez.)
        \crefformat{subsection}{\sffamily#2sez.~#1#3} % Con la minuscola
        \Crefformat{subsection}{\sffamily#2Sez.~#1#3} % Con la maiuscola
    %   \subsubsection (par. e Par.)
        \crefformat{subsubsection}{\sffamily#2par.~#1#3} % Con la minuscola
        \Crefformat{subsubsection}{\sffamily#2Par.~#1#3} % Con la maiuscola
    %   Pagina
        \crefformat{page}{\sffamily#2pag.~#1#3}
        \Crefformat{page}{\sffamily#2Pag.~#1#3}
    % Per riferirsi sia all'oggetto sia alla pagina in cui è presente
        \newcommand{\crefpage}[1]{\cref{#1} \textsf{a} \cpageref{#1}}
% -------[ fine ]-------------------------------------------------------------------

% -------[ PERSONALIZZAZIONE \SECTION, \SUBSECTION E \SUBSUBSECTION ]---------------
    % Definisci il nuovo controllo condizionale \iffirstsubsection per verificare se è la prima subsection della section a cui appartiene
        \newif\iffirstsubsection
            \firstsubsectiontrue
    % Riga orizzontale di separazione (HOrizontal rule)
        \newcommand{\horule}{\noindent\rule{\linewidth}{0.4pt}}
    % \SECTION
        % Ridefinisci le \section affinché si aprano sempre a pagina nuova e, nella pagina in cui si aprono, non ci sia l'intestazione relativa a sezione o sottosezione (lati esterni sinistro e destro)
        \let\ssection\section
        \renewcommand{\section}{
            \clearpage
            \thispagestyle{inizio-section}
            \firstsubsectiontrue % Dico che è iniziata una nuova section, affinché la prima subsection non abbia l'\horule
            \ssection
        }
    % \SUBSECTION
        % Ridefinisci le \subsection affinché, fuorché la prima di ogni section, abbiano tutte la riga di separazione (\horule)
        \let\ssubsection\subsection
        \renewcommand{\subsection}{
            \iffirstsubsection % Se è la prima subsection dopo una nuova section,
                \firstsubsectionfalse % allora avvisa che la prossima sarà la seconda (e adesso non mettere l'\horule
            \else % Se non è la prima subsection dopo una nuova section,
                \horule % allora stampa l'\horule
            \fi
            \ssubsection
        }
    % \SUBSUBSECTION
        % Ridefinisci le \subsubsection affinché siano stilisticamente come dei paragrafi
        \titleformat{\subsubsection}[runin] % [runin] affinché il testo che segue stia sulla stessa riga del titolo della subsection
            {\normalfont\normalsize\bfseries} % Stile del titolo
            {} % Etichetta (niente numero o simbolo prima del titolo)
            {0pt} % Spazio tra l’etichetta e il titolo
            {} % Codice da eseguire prima del titolo (nulla)
            [.] % Codice da eseguire dopo il titolo (metti il punto)
% -------[ fine ]-------------------------------------------------------------------

% -------[ DEFINIZIONE \ESEMPIO ]---------------------------------------------------
    % Manina (per mettere in risalto gli esempi nel testo)
    \newcommand{\Matita}{
        \normalfont\hspace{-0.25em}\raisebox{-0.3em}[0pt][0pt]{\scalebox{1.2}[1.1]{\PencilRight}}\hspace{0.1em}
    }
    % Definisci lo stile specifico del paragrafo "Esempio" (matita)
    %   Comando: \es
    %   \es*[opzionale]{obbligatorio}
    %   *           :  indica se includere l'esempio nel toc e nei segnalibri PDF
    %   opzionale   :  indica il titolo dell'esempio per il toc e i segnalibri PDF
    %   obbligatorio:  indica il titolo dell'esempio visibile nel PDF
        \NewDocumentCommand{\es}{s o m}{
            \IfBooleanTF{#1} {
            %   \es*{lorem}
                {\paragraph*{\Matita{}\itshape #3.}}
            }{
            %   \es{lorem}
                \IfNoValueTF{#2}
                    {\paragraph[#3]{\Matita{}\itshape #3.}}
            %   \es[ipsum]{lorem}
                {\paragraph[#2]{\Matita{}\itshape #3.}}
            }
        }
    % Comando per finire graficamente l'esempio (con l'inserimento di \PencilLeft)
    %   \FineEs       se l'esempio finisce in mathmode (non inline)
    %   \FineEsTesto  se l'esempio finisce con una normale riga di testo
        \newcommand{\FineEs}{\tag*{\raisebox{-0.2em}[0pt][0pt]{\scalebox{1}[1]{\PencilLeft}}}}
        \newcommand{\FineEsTesto}{\hfill\raisebox{-0.2em}[0pt][0pt]{\scalebox{1}[1]{\PencilLeft}}}
% -------[ fine ]-------------------------------------------------------------------

% -------[ DEFINIZIONE \OSSERVAZIONE ]----------------------------------------------
    % Definisci lo stile specifico del paragrafo "Osservazione/Nota bene" (manina)
    %   Comandi: \oss \osservazione \nb \notabene
        \newcommand{\oss}[1]{
            \paragraph[Osservazione]{\normalfont % No grassetto
                \raisebox{-0.1em}[0pt][0pt]{\scalebox{1.2}[1]{\HandRight}}\hspace{-0.4em} % Manina
                {\itshape #1} % Eventuale titolo dell'osservazione tutto in corsivo (punto escluso)
            }
        }
        \newcommand{\osservazione}[1]{\oss{#1}}
        \newcommand{\nb}[1]{\oss{#1}}
        \newcommand{\notabene}[1]{\oss{#1}}
% -------[ fine ]-------------------------------------------------------------------

% -------[ DEFINIZIONI MATEMATICHE PERSONALIZZATE ]---------------------------------
    \newcommand{\co}[2]{\left(#1\,,#2\right)} % Coppia Ordinata \co12 => (1 , 2)
    \renewcommand{\i}{\mathrm i} % Unità immaginaria \i
    \renewcommand{\Re}{\operatorname{Re}} % Parte reale dei numeri complessi \Re
    \renewcommand{\Im}{\operatorname{Im}} % Parte immaginaria dei numeri compessi \Im
    \newcommand{\an}{\left\{a_n\right\}} % Successione a_n \an
    \renewcommand{\d}{\textrm d} % Differeziale d
    
    % Insiemistica
        \newcommand{\Set}[1]{\left\{ #1 \right\}} % Rappresentazione d'insieme per elencazione \Set{1,2,3} => {1,2,3}
        \newcommand{\N}{\mathbb{N}} % Insieme dei numeri naturali
        \newcommand{\Z}{\mathbb{Z}} % Insieme dei numeri relativi
        \newcommand{\Q}{\mathbb{Q}} % Insieme dei numeri razionali
        \newcommand{\R}{\mathbb{R}} % Insieme dei numeri reali
        \newcommand{\C}{\mathbb{C}} % Insieme dei numeri complessi
    
    % Lettere greche preferite (inverto quella di default e quella alternativa)
        % epsilon
            \let\altepsilon\epsilon % ALTernativa (sulla base delle mie pref.)
            \let\corepsilon\varepsilon % CORretta (sulla base delle mie pref.)
            \renewcommand{\epsilon}{\corepsilon}
            \renewcommand{\varepsilon}{\altepsilon}
        % phi
            \let\altphi\phi % ALTernativa (sulla base delle mie pref.)
            \let\corphi\varphi % CORretta (sulla base delle mie pref.)
            \renewcommand{\phi}{\corphi}
            \renewcommand{\varphi}{\altphi}
% -------[ fine ]-------------------------------------------------------------------
%   3. In questo file, andare alla riga
%       pdftitle = {***},
%      e sostituire a *** l'effettivo titolo del documento,
%      affinché il metadato "Titolo" del pdf sia aggiornato correttamente
% -------[ fine ]-------------------------------------------------------------------

% -------[ INDICE DEL PACCHETTO ]---------------------------------------------------
%   IMPOSTAZIONI INIZIALI
%      Definizione dell'autore (mio nome e mail)
%      Impostazione pagina, testo e caratteri
%      Matematica
%      Figure e grafici
%   PERSONALIZZAZIONI
%      Personalizzazione didascalie
%      Personalizzazione intestazioni e piè di pagina
%      Collegamenti e segnalibri
%      Personalizzazione riferimenti
%      Personalizzazione \section, \subsection e \subsubsection
%    DEFINIZIONI PERSONALIZZATE
%      Definizione \esempio
%      Definizione \osservazione
%      Definizioni matematiche personalizzate
% -------[ fine ]-------------------------------------------------------------------

% ----------------------------------------------------------------------------------
% -------[ INIZIO CONTENUTO PACCHETTO ]---------------------------------------------
% -------[ Non toccare se non si capisce cosa c'è scritto ]-------------------------
% ----------------------------------------------------------------------------------

% -------[ DEFINIZIONE DELL'AUTORE (MIO NOME E MAIL) ]------------------------------
    \newcommand{\Autore}{Edoardo~Porcaro}
    \newcommand{\EdoardoPorcaro}{\href{mailto:ufogamespa@gmail.com}{\Autore{}}}
% -------[ fine ]-------------------------------------------------------------------

% -------[ IMPOSTAZIONE PAGINA, TESTO E CARATTERI ]---------------------------------
    \usepackage[utf8]{inputenc} % Codifica input UTF8
    \usepackage[T1]{fontenc} % Codifica output T1 (supporto a lettere accentate europee)
    \usepackage[italian]{babel} % Usa la lingua italiana (sillabazione e nomenclatura sezioni, indici e etichette (es. "figura"))
        \usepackage{silence}\WarningFilter{latex}{Command \showhyphens has changed} % Dalla versione 2025 di pdfLaTeX esce questo warning che, con questo pacchetto e comando, nascondo dal log
    \addto\captionsitalian{
        \renewcommand{\abstractname}{Descrizione del corso} % Rinomina il "nome" dell'abstract
        %\renewcommand{\listfigurename}{Indice delle figure} % Rinomina il "nome" del listoffigures
        \renewcommand{\contentsname}{Indice dei contenuti}} % Rinomina il "nome" del tableofcontents
    \usepackage{microtype} % Rende la tipografia più professionale e raffinata
    \usepackage{multicol} % Per consentire un fine controllo su pagine e sezioni multicolonna
    \usepackage{titlesec} % Usato per ridefinire le \subsubsection nel blocco [Personalizzazione \subsubsection]
    \usepackage{enumitem} % Consente di personalizzare gli elenchi puntati o numerati (listati), modificandone margini, spaziatura, simboli, numerazioni
    \usepackage{pifont} % Per numeretti cerchiati (es. \ding{172} = (1))
    \usepackage{bbding} % Per \PencilRight in \es e \HandRight in \oss
    %\setlength{\parindent}{0pt} % Rimuovi l'indentazione a inizio nuova riga
    \setcounter{secnumdepth}{3} % Numerazione fino al livello \subsubsection [2]
    \setcounter{tocdepth}{3} % Metti nel toc fino al livello \subsubsection [3]
    \frenchspacing % Usa la spaziatura europea dopo i segni di punteggiatura
% -------[ fine ]-------------------------------------------------------------------

% -------[ MATEMATICA ]-------------------------------------------------------------
    \usepackage{amsmath,amssymb,amsfonts,amsthm} % Pacchetti standard dell'AMS
        \numberwithin{equation}{section} % Per avere la numerazione delle equazioni che segua quella delle \section
    \usepackage{mathtools} % Fornisce comandi aggiuntivi per l’allineamento e la formattazione di equazioni (es. \coloneqq)
    \usepackage{mathrsfs} % Per lettere calligrafiche "più corsive" (es. \mathscr{C})
    \usepackage{bbm} % Per rendere qualsiasi carattere (numeri inclusi!) in stile blackboard bold (es. \mathbbm{R} -- che in questo caso è uguale a \mathbb{R})
    \usepackage{cancel}
    \usepackage{stmaryrd} % Introduce simboli matematici aggiuntivi non presenti negli standard LaTeX o AMS (es. \llbracket e \rrbracket)
    \usepackage{siunitx}
% -------[ fine ]-------------------------------------------------------------------

% -------[ FIGURE E GRAFICI ]-------------------------------------------------------
    \usepackage{graphicx}
    \usepackage{tikz}
        \usetikzlibrary{arrows.meta,positioning} % Per opzioni avanzate per le punte delle frecce e per il posizionamento relativo dei nodi
        \usetikzlibrary{angles,quotes} % Per disegnare angoli ed etichette
        \usetikzlibrary{patterns} % Per "escludere" aree del piano con i pattern
        \usetikzlibrary{calc} % Per effettuare calcoli dentro le coordinate
    % Definizione di elementi asse, tratteggio e tratteggio scuro per tutti i grafici TikZ
        \tikzset{
            asse/.style = {-{Stealth[length=6pt,width=6pt]}},
            punto/.style = {circle, fill=black, inner sep=1.2pt},
            punto/.style = {circle, fill=black, minimum size=1.2pt},
            tratteggio/.style = {dashed, gray},
            tratteggioscuro/.style = {dashed, black}
        }
    % Definizioni di stile per i grafici con pgfplots e l'ambiente axis
    \usepackage{pgfplots}
        \pgfplotsset{compat=1.18}

        % Stile di default per tutti gli axis
        \pgfplotsset{
            every axis/.style={
                axis lines=middle,
                xlabel={$x$}, ylabel={$y$},
                xmin=-5, xmax=5,
                ymin=-5, ymax=5,
                xtick={0}, ytick={0},
                unbounded coords=jump,
                enlargelimits=false,
            },
        % Stile di default per tutti gli \addplot
            every axis plot/.style={
                thick,
                smooth,
                samples=200
            }
        }
    % Definizione di punto (come comando stand alone: la definizione in tikzset non si scala correttamente)
        \newcommand{\punto}[1]{\filldraw (#1) circle (1.2pt)}
% -------[ fine ]-------------------------------------------------------------------

% -------[ PERSONALIZZAZIONE DIDASCALIE ]-------------------------------------------
    \usepackage{caption}
        \captionsetup{
            labelfont = sf, % Etichetta (es. "Figura 1") in corsivo 
            labelsep = period, % Separatore "." tra etichetta e testo (es. "Figura 1. Lorem ipsum") (di default è ":")
            textfont = normalfont,% testo normale (puoi mettere it per corsivo)
            %justification=raggedright, % evita giustificazione completa
            %singlelinecheck=false, % applica lo stile anche se la didascalia è su una sola riga
            margin = 1.5cm % Stringi specularmente i margini di 1,5cm di tutta la didascalia
        }
    \usepackage{subcaption}
% -------[ fine ]-------------------------------------------------------------------

% -------[ PERSONALIZZAZIONE INTESTAZIONI E PIÈ DI PAGINA ]-------------------------
    \usepackage{fancyhdr}
    % Carica i nomi di section/subsection da usare nelle intestazioni
        \renewcommand{\sectionmark}[1]{\markboth{\thesection~\,#1}{}}
        \renewcommand{\subsectionmark}[1]{\markright{\thesubsection~\,#1}}
    % Pulisce intestazione e piè di pagina (per modificarli a piacimento)
        \fancyhf{}
    % Imposta l'altezza dell'intestazione affinché sia sufficiente (>13,5 pt)
        \setlength{\headheight}{14pt}
    % Nessuna linea orizzontale né sotto all'intestazione né sopra al piè di pagina
        \renewcommand{\headrulewidth}{0pt}
        \renewcommand{\footrulewidth}{0pt}
    % Definisci lo stile di intestazioni e piè di pagina predefinito
        \fancypagestyle{predefinito}{
            % Lato esterno sinistro (Left Even): "[section]"
                \fancyhead[LE]{\sffamily\nouppercase{\leftmark}}
            % Lato esterno destro (Right Odd): "[subsection]"
                \fancyhead[RO]{\sffamily\nouppercase{\rightmark}}
            % Lati interni (Left Odd e Right Even): "Edoardo Porcaro"
                \fancyhead[LO,RE]{\EdoardoPorcaro{}}
            % Numeri di pagina nei lati esterni del piè di pagina (a sinistra nelle pagine pari e a destra nelle pagine dispari)
                \fancyfoot[LE,RO]{\sffamily\thepage}
        }
    % Definisci lo stile di intestazioni e piè di pagina per le pagine in cui si apre una \section
        \fancypagestyle{inizio-section}{
            % Svuoto i lati esterni sinistro e destro affinché non appaiano indicazioni su sezione o sottosezione
            %   Il resto lo lascio invariato, ossia come definito dallo stile "predefinito"
                \fancyhead[LE,RO]{}
        }
    % Imposta lo stile di intestazioni e piè di pagina predefinito per tutto il documento (salvo eccezioni definite pagina per pagina nel corpo del documento che sovrascrivono lo stile predefinito)
        \pagestyle{predefinito}
% -------[ fine ]-------------------------------------------------------------------

% -------[ COLLEGAMENTI E SEGNALIBRI ]----------------------------------------------
    \usepackage{hyperref}
    % Imposta le proprietà (i metadati) del documento PDF
        \hypersetup{
            linktoc = all, % Pure i numeri di pagina nel toc sono cliccabili
            hidelinks,
            pdftitle = {Riassunto di analisi matematica 1},
            pdfauthor = {\Autore{}},
            pdfsubject = {Analisi matematica 1},
            %pdfkeywords = {keywords},
            pdfcreator = {\Autore{}},
            pdfproducer = {Computer di \Autore{}},
            %pdfpagetransition = {Wipe /Di 270 /SS 0.5}
            pdfdisplaydoctitle = true % Mostra il titolo (pdftitle) del documento nel titolo della finestra del lettore PDF (anzichè il nome del file)
        }
    % Rendi tutti i collegamenti (\href) in sans serif
        \let\oldhref\href
        \renewcommand{\href}[2]{\sffamily\oldhref{#1}{#2}}
    % Metti nei segnalibri del PDF tutti i livelli fino al livello \paragraph [4]
    \usepackage{bookmark}
        \bookmarksetup{depth=4}
% -------[ fine ]-------------------------------------------------------------------

% -------[ PERSONALIZZAZIONE RIFERIMENTI ]------------------------------------------
    \usepackage{varioref} % Per usare \vref, che indica anche la pagina se l'elemento a cui punta non è sulla stessa pagina del comando
    \usepackage{cleveref} % Per usare \cref, che indica anche l'etichetta dell'elemento a cui punta (ad esempio, scrive automaticamente "fig." oltre suo numero)
    %   Traduzioni italiane delle congiunzioni
        \newcommand{\crefpairconjunction}{ e } % Tra due referenze
        \newcommand{\crefmiddleconjunction}{, } % Tra referenze intermedie
        \newcommand{\creflastconjunction}{ e } % Prima dell’ultima referenza
    % Tutti i riferimenti resi in sans serif
    %   Equazioni (eq., eqq., Eq. e Eqq.) (tolgo anche le parentesti altrimenti presenti)
        \creflabelformat{equation}{\sffamily#1}
        % Minuscola
            \crefname{equation}{\sffamily eq.}{\sffamily eqq.}
            \crefrangeformat{equation}{\sffamily eqq. da #1 a #2}
        % Maiuscola
            \Crefname{equation}{Eq.}{Eqq.}
            \Crefrangeformat{equation}{\sffamily Eqq. da #1 a #2}
    %   Figure (fig. e Fig.)
        \creflabelformat{figure}{\sffamily#2#1#3}
            \crefformat{figure}{\sffamily#2fig.~#1#3} % Con la minuscola
            \Crefformat{figure}{\sffamily#2Fig.~#1#3} % Con la maiuscola
    %   Tabelle (tab. e Tab.)
        \crefformat{table}{\sffamily#2tab.~#1#3} % Con la minuscola
        \Crefformat{table}{\sffamily#2Tab.~#1#3} % Con la maiuscola
    %   \section (cap. e Cap.)
        \crefformat{section}{\sffamily#2cap.~#1#3} % Con la minuscola
        \Crefformat{section}{\sffamily#2Cap.~#1#3} % Con la maiuscola
    %   \subsection (sez. e Sez.)
        \crefformat{subsection}{\sffamily#2sez.~#1#3} % Con la minuscola
        \Crefformat{subsection}{\sffamily#2Sez.~#1#3} % Con la maiuscola
    %   \subsubsection (par. e Par.)
        \crefformat{subsubsection}{\sffamily#2par.~#1#3} % Con la minuscola
        \Crefformat{subsubsection}{\sffamily#2Par.~#1#3} % Con la maiuscola
    %   Pagina
        \crefformat{page}{\sffamily#2pag.~#1#3}
        \Crefformat{page}{\sffamily#2Pag.~#1#3}
    % Per riferirsi sia all'oggetto sia alla pagina in cui è presente
        \newcommand{\crefpage}[1]{\cref{#1} \textsf{a} \cpageref{#1}}
% -------[ fine ]-------------------------------------------------------------------

% -------[ PERSONALIZZAZIONE \SECTION, \SUBSECTION E \SUBSUBSECTION ]---------------
    % Definisci il nuovo controllo condizionale \iffirstsubsection per verificare se è la prima subsection della section a cui appartiene
        \newif\iffirstsubsection
            \firstsubsectiontrue
    % Riga orizzontale di separazione (HOrizontal rule)
        \newcommand{\horule}{\noindent\rule{\linewidth}{0.4pt}}
    % \SECTION
        % Ridefinisci le \section affinché si aprano sempre a pagina nuova e, nella pagina in cui si aprono, non ci sia l'intestazione relativa a sezione o sottosezione (lati esterni sinistro e destro)
        \let\ssection\section
        \renewcommand{\section}{
            \clearpage
            \thispagestyle{inizio-section}
            \firstsubsectiontrue % Dico che è iniziata una nuova section, affinché la prima subsection non abbia l'\horule
            \ssection
        }
    % \SUBSECTION
        % Ridefinisci le \subsection affinché, fuorché la prima di ogni section, abbiano tutte la riga di separazione (\horule)
        \let\ssubsection\subsection
        \renewcommand{\subsection}{
            \iffirstsubsection % Se è la prima subsection dopo una nuova section,
                \firstsubsectionfalse % allora avvisa che la prossima sarà la seconda (e adesso non mettere l'\horule
            \else % Se non è la prima subsection dopo una nuova section,
                \horule % allora stampa l'\horule
            \fi
            \ssubsection
        }
    % \SUBSUBSECTION
        % Ridefinisci le \subsubsection affinché siano stilisticamente come dei paragrafi
        \titleformat{\subsubsection}[runin] % [runin] affinché il testo che segue stia sulla stessa riga del titolo della subsection
            {\normalfont\normalsize\bfseries} % Stile del titolo
            {} % Etichetta (niente numero o simbolo prima del titolo)
            {0pt} % Spazio tra l’etichetta e il titolo
            {} % Codice da eseguire prima del titolo (nulla)
            [.] % Codice da eseguire dopo il titolo (metti il punto)
% -------[ fine ]-------------------------------------------------------------------

% -------[ DEFINIZIONE \ESEMPIO ]---------------------------------------------------
    % Manina (per mettere in risalto gli esempi nel testo)
    \newcommand{\Matita}{
        \normalfont\hspace{-0.25em}\raisebox{-0.3em}[0pt][0pt]{\scalebox{1.2}[1.1]{\PencilRight}}\hspace{0.1em}
    }
    % Definisci lo stile specifico del paragrafo "Esempio" (matita)
    %   Comando: \es
    %   \es*[opzionale]{obbligatorio}
    %   *           :  indica se includere l'esempio nel toc e nei segnalibri PDF
    %   opzionale   :  indica il titolo dell'esempio per il toc e i segnalibri PDF
    %   obbligatorio:  indica il titolo dell'esempio visibile nel PDF
        \NewDocumentCommand{\es}{s o m}{
            \IfBooleanTF{#1} {
            %   \es*{lorem}
                {\paragraph*{\Matita{}\itshape #3.}}
            }{
            %   \es{lorem}
                \IfNoValueTF{#2}
                    {\paragraph[#3]{\Matita{}\itshape #3.}}
            %   \es[ipsum]{lorem}
                {\paragraph[#2]{\Matita{}\itshape #3.}}
            }
        }
    % Comando per finire graficamente l'esempio (con l'inserimento di \PencilLeft)
    %   \FineEs       se l'esempio finisce in mathmode (non inline)
    %   \FineEsTesto  se l'esempio finisce con una normale riga di testo
        \newcommand{\FineEs}{\tag*{\raisebox{-0.2em}[0pt][0pt]{\scalebox{1}[1]{\PencilLeft}}}}
        \newcommand{\FineEsTesto}{\hfill\raisebox{-0.2em}[0pt][0pt]{\scalebox{1}[1]{\PencilLeft}}}
% -------[ fine ]-------------------------------------------------------------------

% -------[ DEFINIZIONE \OSSERVAZIONE ]----------------------------------------------
    % Definisci lo stile specifico del paragrafo "Osservazione/Nota bene" (manina)
    %   Comandi: \oss \osservazione \nb \notabene
        \newcommand{\oss}[1]{
            \paragraph[Osservazione]{\normalfont % No grassetto
                \raisebox{-0.1em}[0pt][0pt]{\scalebox{1.2}[1]{\HandRight}}\hspace{-0.4em} % Manina
                {\itshape #1} % Eventuale titolo dell'osservazione tutto in corsivo (punto escluso)
            }
        }
        \newcommand{\osservazione}[1]{\oss{#1}}
        \newcommand{\nb}[1]{\oss{#1}}
        \newcommand{\notabene}[1]{\oss{#1}}
% -------[ fine ]-------------------------------------------------------------------

% -------[ DEFINIZIONI MATEMATICHE PERSONALIZZATE ]---------------------------------
    \newcommand{\co}[2]{\left(#1\,,#2\right)} % Coppia Ordinata \co12 => (1 , 2)
    \renewcommand{\i}{\mathrm i} % Unità immaginaria \i
    \renewcommand{\Re}{\operatorname{Re}} % Parte reale dei numeri complessi \Re
    \renewcommand{\Im}{\operatorname{Im}} % Parte immaginaria dei numeri compessi \Im
    \newcommand{\an}{\left\{a_n\right\}} % Successione a_n \an
    \renewcommand{\d}{\textrm d} % Differeziale d
    
    % Insiemistica
        \newcommand{\Set}[1]{\left\{ #1 \right\}} % Rappresentazione d'insieme per elencazione \Set{1,2,3} => {1,2,3}
        \newcommand{\N}{\mathbb{N}} % Insieme dei numeri naturali
        \newcommand{\Z}{\mathbb{Z}} % Insieme dei numeri relativi
        \newcommand{\Q}{\mathbb{Q}} % Insieme dei numeri razionali
        \newcommand{\R}{\mathbb{R}} % Insieme dei numeri reali
        \newcommand{\C}{\mathbb{C}} % Insieme dei numeri complessi
    
    % Lettere greche preferite (inverto quella di default e quella alternativa)
        % epsilon
            \let\altepsilon\epsilon % ALTernativa (sulla base delle mie pref.)
            \let\corepsilon\varepsilon % CORretta (sulla base delle mie pref.)
            \renewcommand{\epsilon}{\corepsilon}
            \renewcommand{\varepsilon}{\altepsilon}
        % phi
            \let\altphi\phi % ALTernativa (sulla base delle mie pref.)
            \let\corphi\varphi % CORretta (sulla base delle mie pref.)
            \renewcommand{\phi}{\corphi}
            \renewcommand{\varphi}{\altphi}
% -------[ fine ]-------------------------------------------------------------------
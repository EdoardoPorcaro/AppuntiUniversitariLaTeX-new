%\begin{figure}\centering
%\begin{tikzpicture}
%\begin{axis}[
    %axis lines=middle,       % assi incrociati al centro
    %xlabel={$x$}, ylabel={$y$},
    %xmin=-5, xmax=5,
    %ymin=-5, ymax=5,
    %xtick={0},     % tacche sugli assi
    %ytick={0},
    %unbounded coords=jump,
    %grid=both,               % griglia opzionale
    %width=9cm, height=9cm, % dimensioni
    %enlargelimits=false
%]
    % Funzione da tracciare
    %\addplot[thick, samples=200, smooth, domain=-5:-2.01] {(x+1)/(x+2)};
    %\addplot[thick, samples=200, smooth, domain=-1.99:5] {(x+1)/(x+2)};
    %\addplot[dashed, samples=2, smooth, domain=-2.01:-1.99] {(x+1)/(x+2)};
    %\addplot[dashed, samples=200, smooth, domain=-5:5] {-1};
    %\addplot[black, mark=*] coordinates {(2,3)}
        %node[above right] {$A$};
%\end{axis}
%\end{tikzpicture}\end{figure}

\begin{figure}
    \centering
    \begin{tikzpicture}
        % Estensione assi
        \def\X{2.5} % Estensione asse ascisse (4 per grafico singolo, 2.5 per grafici multipli)
        \def\Y{2.25} % Estensione asse ordinate (3 per grafico singolo, 2.25 per grafici multipli)
        
        % Assi e origine
        \draw[asse] (-\X,0)--(\X,0);
            \node[below,font=\small] at (96/100*\X,0) {\(x\)};
        \draw[asse] (0,-\Y)--(0,\Y);
            \node[left,font=\small] at (0,96/100*\Y) {\(y\)};
        \node[below left, font=\small] at (0,0) {\(0\)};

        % Funzione
        \draw[domain=-\X:\X,smooth,samples=400,variable=\x,black,thick]
        plot ({\x}, {\x+3});
    \end{tikzpicture}
    \caption[]{}
    \label{}
\end{figure}